%---------------------------------------------------------------------------------------------------
% WMS PERSONNEL MGMT CATEGORIES AND THE DEFINITION OF "STRUCTURED"
%
% In the WMS scoring guide, a score of 3 and above implies that there are at least some formal structures in place, while a score of 2 and below implies that while there may be a process in place, it is entirely informal and it would not be carried out if the individual manager who led it was not present. 
%
% 1.Instilling a talent mindset/ Managing Talent <--THIS IS THE MAIN THING
% 	a) How do senior managers show that attracting and developing talent is a top priority?
% 	b) Do senior managers get any rewards for bringing in and keeping talented people in the company?
% 2. Building a High-Performance Culture through Incentives and Appraisals
% 	a) How does your appraisal system work? Tell me about the most recent round?
% 	b) How does the bonus system work?
% 	c) Are there any non-financial rewards for top performers?
% 	d) How does your reward system compare to your competitors?
% 3. Removing Poor Performers/ Making Room for Talent
% 	a) If you had a worker who could not do his job what would you do? Could you give me a recent example?
% 	b) How long would underperformance be tolerated?
% 	c) Do you find any workers who lead a sort of charmed life? Do some individuals always just manage to avoid being fixed/fired?
% 4. Developing Talent and Promoting High-Performers
% 	a) Tell me about your promotion system.
% 	b) What about poor performers? What happens with them? Are there any examples you can think of?
% 	c) How would you identify and develop your star performers?
% 	d) If two people both joined the company 5 years ago and one was much better than the other what job opportunities would he/she have in the company?
% 5. Distinctive Employee Value Proposition
% 	a) What makes it distinctive to work at your company as opposed to your competitors?
% 	b) If you were trying to sell your firm to me how would you do this (get them to try to do this)?
% 	c) What don’t people like about working in your firm?
% 6. Retaining Talent
% 	a) If you had a star performer who wanted to leave what would the company do?
% 	b) Could you give me an example of a star performers being persuaded to stay after wanting to leave?
% 	c) Could you give me an example of a star performer who left the company without anyone trying to keep them?
%---------------------------------------------------------------------------------------------------
% Last updated:
% 7 Apr 19 - CC
%!TEX root=MGMT.tex

% 1. Set-up - building a productive workforce
% Not sure about the vague reference to information and agency problems - be more specific or drop. Note: {Hoffman:Discretion:QJE:2018} points to manager bias in the face of objective measure of skill, emphasizing the importance for management practices to promote and reward the use of the right information.
Workforce productivity is a key driver of competitiveness. Building a productive workforce requires hiring and retaining the right employees, while removing poor performers. Hiring, retention and dismissal decisions represent the core of personnel management, but are fraught with information and agency problems. How to deal with these problems and execute these decisions, to a large degree, remains an open question. Addressing it is essential for understanding a range of important labor-market phenomena, such as increasing inequality, the value of bosses, the rise in managerial pay and the evolution of ``superstar firms''.

% Workforce productivity is a key driver of competitiveness. Building a productive workforce, however, is a non-trivial organizational challenge fraught with information and agency problems. Hiring, retention and dismissal decisions are at the core of personnel management strategies, and good firms recognize the underlying matching problem that different skills are required at different levels in the organization. Understanding the relationship between personnel management practices, labor matching processes and productivity is essential for addressing larger questions related to productivity and pay differences, such as increasing inequality, the value of bosses, the rise in managerial pay and of ``superstar firms''.

% 2. Why the question remains open and overcoming the obstacle
A persistent obstacle to examining how personnel management leads to a productive workforce is the inability to combine the necessary detailed information about firms and their employees with empirical measures of their management practices. In particular, we need data on organizational structure (distinguishing managerial and production layers), worker mobility (identifying new hires, internal promotions and separations), human capital (measuring experience and education) and measures of personnel management practices. We overcome this obstacle by constructing an empirical setting, anchored in administrative data from Brazil, that uniquely meets these requirements.

% 3. Empirical setting overview
Our empirical setting links employer-employee matched data from Brazil's \emph{Rela\c{c}\~{a}o Anual de Informa\c{c}\~{o}es Sociais} (RAIS) to the Brazilian manufacturing firms covered in the World Management Survey (WMS). RAIS is different from other large, administrative surveys in two significant ways. First, it explicitly distinguishes managers from production workers, allowing us to connect personnel practices with those employees primarily responsible for carrying them out. Second, it provides the information necessary to identify new hires, internal promotions, and separations, including the reasons for separation. 

The WMS provides our measures of management practices.  Initiated in 2002, the WMS project generates consistent, empirical measures of management practices in establishments across countries \citep{bloom_qje2007}.  Brazil is one of the thirty-five countries included and a sample of its manufacturing firms were surveyed in 2008 and 2013. The WMS uses an interview-based evaluation tool covering 18 basic management practices, a third of which relate specifically to the ``people'' or personnel management practices that directly target hiring, retention and dismissal decisions.  Using the WMS data, we characterize firms as having \emph{structured} or \emph{unstructured} practices, depending on whether they have formal processes in place to handle a particular managerial issue.  

% DS: Should we perhaps be stating the last paragraph as a question? It looks like that's our "problem statement" but it isn't super question-punchy yet.

% People to cite: {Hoffman, Li, Khan}, {Friedrich}, {Christian}

% Alternatively, could we start this paragraph with a strong question of what our paper is about to start building a story? For example, what about this for a question?

% We need an "in this paper" statement, but something that sells it a bit better than just "novel set of stylized facts"

% Given the information and agency challenges faced by managers who have discretion over personnel policy \citep{Hoffman:Discretion:QJE:2018}, the mechanisms behind how different organizational practices --- or, management technology --- help firms assemble a productive workforce is still an open question. Developing a full picture understanding of these relationships requires data on organizational hierarchy (distinguishing managerial and production layers), worker mobility (identifying new hires, internal promotions and separations), human capital (measuring experience and education) and internal managerial policies and processes (measuring management structures). We construct an ideal empirical setting by linking three unique datasets: the full roster of formal employment in Brazil (RAIS), the Brazilian annual industrial survey (PIA) and detailed management practices data from the World Management Survey (WMS). %In this paper, we present a new set of stylized facts that allow for an in-depth comparison of structured management practices at the firm level, manager and worker quality, and their relative contribution to explaining productivity differences across firms. 

% 3. We we do
Using our linked RAIS-WMS data, we provide new insights on how management practices lead to workforce productivity by attracting and retaining high-quality workers at both the manager and production levels of the organization.  First, we evaluate the quality of new hires, incumbents and dismissals for firms with and without structured personnel management practices. Second, we quantify the relationship between structured management practices and manager and production-worker quality. We highlight the contribution of personnel practices, but consider the role of non-personnel (``operations'') practices as well.  

We start by assembling a panel of workers and firms from the 2003–2013 waves of RAIS. The panel delivers our measure of worker quality, which we take as the worker effect in the well-known Abowd-Kramarz-Margolis (AKM) decomposition  of log wages \citep{Abowd1999}. We aggregate our quality measure to the firm level, distinguishing managers from production workers, combine it with the estimated firm effects from the AKM decomposition and firm-year demographic characteristics, and create a firm-level panel that we match to the WMS. Adding input and output data from Brazil's Pesquisa Industrial Anual (PIA) to our RAIS-WMS sample, we then document that, conditional on management practices (and all other factor inputs), firms with higher revenues indeed employ higher quality workers. Although managers comprise only four percent of the workforce in the typical firm, the effect of their quality on revenue is twice that of production workers'.

% This is in line with the findings in \citet{Bender:Management:NBER:2016} for Germany. 
% Although in Brazil managers comprise only four percent of the workforce in the typical firm, their effect on sales is twice as large as that of production workers. It is striking how similar the overall patterns are, given the differences between Brazilian and German industries. Furthermore, over time, firms in the top tercile of WMS management scores capture a larger employment share at the expense of the bottom tercile of scores. 
% We build on work from \citet{Bender:Management:NBER:2016} using German data, where, absent occupation information, they classify workers as managers if they placed in the top quartile of the pay distribution. Because we can more cleanly identify managers and production workers based on detailed occupation codes, we can explore the distribution of workers across the full wage distribution between occupations and firms and develop a better understanding of the mechanisms. 
%\footnote{In German WMS sample, managers account for an average of 8.5\% of the firm's workforce.}

% 4. What we find
Our findings can be summarized as follows. First, firms with structured personnel management practices hire disproportionately from the top half of the worker quality distribution, especially at the production level. Second, these firms are far more successful at keeping their high-quality hires. Their incumbent managers are twice as likely to be high-quality and their advantage among incumbent production workers is almost as great. Third, firms with structured practices have much lower overall dismissal rates and their dismissals are more selective with respect to worker quality.  Finally, we show that the singularly most important personnel practice for attracting and retaining a high-quality workforce is ``instilling a talent mindset'', where hiring high-quality workers is expressly a top priority for the firm and senior managers are rewarded accordingly.  This is true for both production workers and managers. However, we also find that operations management practices are relatively more important for explaining variation in manager quality, suggesting that well-defined targets, accountability and performance evaluation facilitate matches with better managerial talent.  

%We show that firms with structured personnel management practices employ a greater share of high-quality workers than firm that lack structured practices. Furthermore, the incumbent managers in structured firms are twice as likely to be from the top quintile of the manager quality distribution and their advantage among incumbent production workers is almost as great. This advantage is consistently reinforced in hiring and firing. The median manager-level hire in a firm with structured management practices comes from the 60th percentile of the distribution of all hires; firms with unstructured practices get their median manager-level hire from the 48th percentile. The pattern is similar when considering production workers. Firms with structured management hire disproportionately high-quality production workers, while firms with unstructured management practices seem to hire almost at random. Firms with structured management are also much less likely to fire workers at all. When they do, they fire more selectively with respect to worker quality. 

%% QUESTION: What happens to the workers who move from labourer to management, or from management to other? Are we only using observations that are managers throughout or workers throughout?

% 5. Position in the literature/related studies
% Not sure all of these references are really germane.
Our paper bridges the in labor economics focused on the role of worker sorting across firms with the literature in organizational economics that is concerned with recruitment and retention. The labor literature highlights the importance of worker-firm matching for understanding inequality \citep{Card2013,Alvarez:Firms:AEJM:2018,Song:Firming:QJE:2018}, gender wage gaps \citep{Card:Bargaining:QJE:2016}, compensating differentials \citep{Lavetti:CDEM:WP:2018,Sorkin:Ranking:QJE:2018}, and productivity \citep{Iranzo2008}. However, these studies generally abstract away from the processes firms use to assemble their workforcs. The organizational economic literature, on the other hand, emphasizes the importance of recruiting and retention practices \citep{OyerSchaefer:HLE:2011,Hoffman:Discretion:QJE:2018}.  \citet{Hensvik2018} show that firms actively manage workgroups to retain high output, and \citet{shaw_bosses} find that managers and supervisors substantially affect the productivity of their teams. \citet{Jager2016} and \citet{Gallen2018} demonstrate the costliness of going to market to handle turnover.

% The connection between these two pars is weak.

Despite a large conceptual and theoretical body of work on ``what managers do'' \citep{gibbonshenderson_2012}, there is still little empirical evidence on how the practices they follow affect workforce quality. \citet{Bender:Management:JOLE:2018} link WMS data for German firms to administrative earnings records to study how management practices and employee ability are related to productivity.   Using Danish data, \citep{Friedrich2017}  exploits job-to-job transitions and occupational switching to show that more productive firms tend to hire managers internally and select high-quality candidates in terms of their human capital and skill levels. \citet{Bandiera:Matching:JOLE:2015} use Italian administrative data for a random sample of service sector executives to explore how managers and firms match through high- and low-powered incentives.  However, each of these papers lacks some important element of our unique empirical setting -- either the ability to distinguish managers from production workers, identify the reason for separation or empirically measure managerial practices.

%Our paper bridges the literature in organizational economics studying how firms recruit and retain workers, with the literature in labor economics focused on the role of worker sorting across firms. The labor literature highlights the importance of worker-firm matching for understanding inequality \citep{Card2013,Alvarez:Firms:AEJM:2018,Song:Firming:QJE:2018}, gender wage gaps \citep{Card:Bargaining:QJE:2016}, compensating differentials \citep{Lavetti:CDEM:WP:2018,Sorkin:Ranking:QJE:2018}, and productivity \citep{Iranzo2008}. 
% While these studies generally abstract away from the process by which firms actually assemble their workforce, <-- DO "THESE STUDIES" REFER TO THE PRECEDING OR FOLLOWING OR BOTH?
%\citet{Hensvik2018} show that firms actively manage workgroups to retain high output, and \citet{shaw_bosses} estimate that managers and supervisors substantially affect the productivity of their teams. Further, there is evidence that turnover is costly: \citet{Jager2016} and \citet{Gallen2018} show evidence that this is the case because it is difficult to replace incumbent workers with workers from the external labor market. This empirical work is consistent with the recent emphasis in organizational economics on understanding the value of recruiting and retention practices \citep{OyerSchaefer:HLE:2011,Hoffman:Discretion:QJE:2018}. 

% The empirical connection between overall management practices and firm productivity has been well-established \citep{bloom_india2012, wms_jeea}. However, despite a large conceptual and theoretical body of work on ``what managers do'' \citep{gibbonshenderson_2012}, we still have little empirical evidence on whether and how structured personnel management practices affect workforce quality. Our paper is most closely related to  \citet{Bandiera:Matching:JOLE:2015} and especially \citet{Bender:Management:JOLE:2018}. Both papers combine survey data on firm management practices with administrative data that can measure hiring and compensation outcomes. \citet{Bandiera:Matching:JOLE:2015} use Italian administrative data for a random sample of service sector executives to show a rich characterization of how managers and firms match through high- and low-powered incentives. \citet{Bender:Management:JOLE:2018} link WMS data for German firms to administrative earnings records to study how management practices and employee ability are related to productivity. Our work is also related to \citet{Friedrich2017}, who shows that more productive firms tend to hire managers internally and also select high-quality candidates in terms of their human capital and skill levels. However, none of the papers in the literature focus directly on personnel management, nor are able to directly distinguish managerial and non-managerial workers and characterize the relationship between these factors as we are able to do with our unique empirical setting.

% I'm really not sure how to weave this in now. \footnote{There is some evidence that better management is associated with more efficient use of energy inputs \citep{Boyd:Evidence:JEEM:2014}, though not much on other types of inputs such as labor inputs.} 
% There is a large literature related to personnel policy and productivity, though there is still a gap in understanding how structured management practices may help alleviate the information and agency problems inherent in the labour matching process. \citep{Oyer and Schaeffer 2011} We build on earlier work by \citet{Bender:Management:NBER:2016}, where they match WMS data for German firms to earnings collected by Germany's Institute for Employment Research and examine how management practices and employee ability are related to productivity. They find that better managed firms are more productive, but a significant portion of the effect of management quality is explained by the human capital of the highest-paid workers in the firm. Two important limitations of the German data, however, are that their records do not include occupation codes (to cleanly distinguish between production workers and managers) nor the reasons for separation (to distinguish fires from quits). Using detailed Danish data, \citep{Friedrich2017}  exploits job-to-job transitions and occupational switching and finds that more productive firms tend to hire managers internally and select high-quality candidates in terms of their human capital and skill levels. The Danish setting, however, lacks data on the level of adopted managerial structures and also does not record the reason for a job separation. 

% \textbf{This paragraph is clunky. I'm leaving it for Ian to do his magic with lit knowledge powers.}
%  Our work also adds to the literature on sorting and matching. There is evidence that assortative matching is a key driver of inequality in the US, Brazil and Germany, \citep{???} though it is unclear what the specific mechanisms behind this assortative matching are within and across firms. There is evidence that more productive (and better-paying) firms employ more high quality (generally better paid) workers. \citep{CHK; Engbom and Moser; Alvarez et al.} There is also evidence that workers sort both on the basis of wage as well as non-wage aspects of the job \citet{Sorkin; Lavetti and Schmutte; Hotz et al.}, though we rarely observe non-wage aspects of the job (note, here is where we could do a nod to our result on operations management explaining variation in the manager quality).  

