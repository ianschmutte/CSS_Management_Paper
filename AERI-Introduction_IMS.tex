% Last updated:
% 14 Jan 19 - CC
%!TEX root=AER_Insights.tex
% 1. Set-up - Understanding the relationship between talent management, employee matching and productivity 
Workforce quality is a key driver of productivity. 
Building a high-quality workforce, however, is a non-trivial organizational challenge fraught with information and agency problems. To address these problems, firms develop processes for personnel management that govern how they choose whom to hire, retain, and fire. 
% Ideally, personnel management processes help the firm's managers to recognize quality, while creating proper incentives to retain and motivate the best workers. 
If it is very difficult to get managers to identify worker quality and assemble and retain a high-quality workforce, then the considerable differences across firms in productivity and pay are partially explained by differences in the effectiveness of their personnel management practices. 
Understanding the relationship between personnel management practices, employee matching and productivity is essential for addressing larger questions related to productivity and pay differences, such as increasing inequality, the rise in managerial pay and of ``superstar firms''. 


As economists exploit the increasing availability of firm-level administrative data, we have a large and growing body of research documenting considerable dispersion across firms in productivity, compensation, and worker quality. 




 
