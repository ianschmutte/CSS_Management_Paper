% Last updated:
% 14 Jan 19 - CC
%!TEX root=MGMT.tex
% 1. Set-up - Understanding the relationship between talent management, employee matching and productivity 
Workforce productivity is a key driver of organizational competitiveness.  
Building a productive workforce involves hiring and retaining the right employees, while shedding those who do not measure up. Any personnel management strategy responsible for hiring, retention and turnover must recognize the heterogeneity in the underlying matching problem -- different skills are required at different levels in the organization. Understanding the relationship between personnel management, employee matching and productivity is essential for addressing larger questions related to productivity and pay differences, such as increasing inequality and the rise of so-called ``superstar firms''.  
% From McKinsey, An agenda for the talent-first CEO
% Thirty-seven people in a 12,000-employee company! [Blackstone] In almost every organization, success depends on a small core of people who deliver outsize value. The success of the talent-first CEO largely depends on how he or she leverages this critical 2 percent of people. (That 2 percent figure is merely a guideline; in big corporations, the “2 percent” may be a group of fewer than 200 people.)

% 2. Data requirements to develop the understand and our unique setting
Developing such an understanding requires detailed information about firms, their employees and their management practices. In particular, we need data on organizational structure (distinguishing managerial and production layers), worker mobility (identifying new hires, internal promotions and separations), human capital (measuring experience and education) and processes for personnel management.  In this paper, we construct an empirical setting that uniquely meets these requirements by linking employer-employee matched data from Brazil's \emph{Rela\c{c}\~{a}o Anual de Informa\c{c}\~{o}es Sociais} (RAIS) to the Brazilian manufacturing firms covered in the World Management Survey (WMS). RAIS is effectively an annual census of all formal-sector jobs in Brazil, which annually records an extensive range of establishment, job and worker characteristics. Initiated in 2002, the WMS project generates consistent, empirical measures of management practices in establishments across countries (\citep{Bloom:NewEmpirical:WP:2014}).  Brazil is one of the thirty-five countries included and a sample of its manufacturing firms were surveyed in 2008 and 2013. 

% 3. Measuring worker quality and creating the RAIS/WMS sample
We start by assembling a panel of workers and establishments from the 2003–2013 waves of RAIS.  Using the RAIS panel, we construct the well-known Abowd-Kramarz-Margolis (AKM) decomposition  of log wages \citep{Abowd1999} into components associated with time-varying worker characteristics, a time-invariant firm effect and a time-invariant worker effect. Our primary interest is in the worker-specific component of pay, which we treat as a measure of worker quality.  We aggregate our worker-quality measure to the firm level, distinguishing managers from production workers, combine it with the estimated firm effects and firm-year demographic characteristics, and create a firm-level panel we can match to the WMS. 

% 4. Research questions and findings
We exploit our RAIS-WMS linked sample to produce new insights on the importance of personnel management for workforce productivity. First, we document the positive relationship between firm productivity and worker quality.  Using input and output data on the WMS manufacturing firms from Brazil's Pesquisa Industrial Anual (PIA), we find that, conditional on management practices, firms with higher sales indeed employ higher quality workers. While managers comprise only four percent of the workforce in the typical firm, their effect on sales is twice as large as that of production workers. % Insert finding about relative employment growth?

Then, we turn our attention to the role personnel management plays in building a productive workforce. We show that firms with structured personnel management practices employ a greater share of high-quality workers than firms without them.  Their incumbent managers are twice as likely to be high-quality and their advantage among incumbent production workers is almost as great. This advantage is consistently reinforced in hiring and firing. The median manager-level hire in a structured-practices firm comes from the 60th percentile of the manager quality distribution; firms with unstructured practices get their median manager-level hire from the XX percentile. The corresponding gap in production-worker hires is about XX percentage points. Structured-management firms are also more effective in firing low-quality employees, but their edge in separations is more pronounced at production-worker level. 

Finally, we demonstrate that structured personnel management practices and worker quality go hand-in-hand.  Our results suggest that the key personnel practice is ``instilling a talent mindset''. According to the WMS, firms with high talent-mindset ratings make hiring high-quality workers a top priority and reward senior managers accordingly. We find that having a talent mindset predicts better manager and production worker quality, with the larger effect on production workers.

% Conceptual framework that resolves these findings?

% 5. Position in the literature/related studies
While our empirical setting is unique and the insights it facilitates are new, our interest in the relationship between management practices and productivity is shared by a number of recent studies. The most closely related to our work is 
% \citep{Bender:Management:NBER:2016} % ***UPDATE TO JOLE CITE
\citet{Bender:Management:NBER:2016}, who match German WMS data to earnings collected by Germany's Institute for Employment Research to examine how management practices and employee ability are related to productivity. They find that better managed firms are more productive, but a significant portion of the effect of management quality is explained by the human capital of the highest-paid workers in the firm. Their results also indicate that better managed firms employ workers with higher average human capital in general. Unfortunately, the German data do not separate workers by managerial and production levels or distinguish reasons for separation.

Our work is also closely related to \citep{Friedrich2017}, % ***ADD TO BIBTEX FILE 
who uses matched employer-employee data from Denmark to investigate the role internal labor markets play in the recruitment and retention of talent. He exploits job-to-job transitions and occupational switching to document firm differences in internal versus external hiring of managers. Friedrich finds that more productive firms tend to hire managers internally and select high-quality candidates in terms of their human capital and skill levels. As in the German data,  the reasons for worker separations in the Danish data are also missing. More significantly, the Danish setting misses a link to the WMS, which precludes a direct examination of management practices.   

% 6. Broader implications 


