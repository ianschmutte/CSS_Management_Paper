%!TEX root=MGMT.tex

We use a unique set of datasets to explore how organizational structures in a firm relate to the ability of their managers to build a good workforce. While there is evidence that managers do not always make the best decisions on personnel policy, we show the most comprehensive evidence to date that managers in firms that adopt structured management practices make better decisions relative to their counterparts in firms with unstructured management practices. 

For almost 700 firms in Brazil, our dataset includes a decade of job transitions (including wage and occupation information for all formal employees over the period), production and value added data from industrial survey records, and the most detailed data on structured management practices available from the World Management Survey. We use a standard AKM wage decomposition to estimate a firm-specific and a worker-specific effect. The firm-specific effect measures the wage premium that a particular firm offers as a worker moves between firms. Using detailed occupation codes, we separately estimated worker effects for production workers and managers, essentially measuring the value of the worker's portable skills as they move from one job to another.

Based on the scoring methodology of the World Management Survey, we classified firms into using structured management practices or unstructured management practices. While this measure is coarser than the usual standardized measures used in previous work, it is a more intuitive way to think about the differences between their internal organizational practices in this context.   

We find that, consistent with previous work, more productive firms are associated with better quality managers and production workers, as well as more structured management practices. With our data, we shed new light on the mechanisms behind the patterns that have been consistently observed across countries. Our results suggest that the advantage of firms that have adopted structured management practices comes from being able to (a) hire more often from the top of the distribution of production worker and manager quality, (b) retain a larger share of high quality workers over time, and (c) make fewer mistakes in selecting workers to dismiss relative to firms with unstructured management practices.

More specifically, we find that more structured personnel practices are particularly important in explaining variation of the quality of production workers in the firm, while it is structured operations practices that are important when explaining the variation in manager quality. This is an intuitive result that highlights the importance of understanding the heterogeneity of the matching problem across different levels of the organization. This is the first step in an exciting research agenda, as it opens the possibility of learning more about the transmission of practices across firms via job-to-job transitions, patterns of gender and race discrimination within and across firms, as well as the highly unequal distribution of pay between levels of the organizations. 

% From McKinsey, An agenda for the talent-first CEO: Thirty-seven people in a 12,000-employee company! [Blackstone] In almost every organization, success depends on a small core of people who deliver outsize value. The success of the talent-first CEO largely depends on how he or she leverages this critical 2 percent of people. (That 2 percent figure is merely a guideline; in big corporations, the “2 percent” may be a group of fewer than 200 people.)
