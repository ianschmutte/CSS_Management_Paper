% Last updated:
% 14 Jan 19 - CC
%!TEX root=MGMT.tex
% 1. Set-up - Understanding the relationship between talent management, employee matching and productivity 
Workforce productivity is a key driver of competitiveness. Building a productive workforce, however, is a non-trivial organizational challenge fraught with information and agency problems. Hiring, retention and dismissal decisions are at the core of personnel management strategies, and good firms recognize the underlying matching problem that different skills are required at different levels in the organization. Understanding the relationship between personnel management practices, labor matching processes and productivity is essential for addressing larger questions related to productivity and pay differences, such as increasing inequality, the value of bosses, the rise in managerial pay and of ``superstar firms''.  

% DS: Should we perhaps be stating the last paragraph as a question? It looks like that's our "problem statement" but it isn't super question-punchy yet.

% From McKinsey, An agenda for the talent-first CEO: Thirty-seven people in a 12,000-employee company! [Blackstone] In almost every organization, success depends on a small core of people who deliver outsize value. The success of the talent-first CEO largely depends on how he or she leverages this critical 2 percent of people. (That 2 percent figure is merely a guideline; in big corporations, the “2 percent” may be a group of fewer than 200 people.)

%People to cite: {Hoffman, Li, Khan}, {Friedrich}, {Christian}

% Alternatively, could we start this paragraph with a strong question of what our paper is about to start building a story? For example, what about this for a question?

Given the information and agency challenges faced by managers who have discretion over personnel policy \citep{Hoffman:Discretion:QJE:2018}, the mechanisms behind how different organizational practices --- or, management technology --- help firms assemble a productive workforce is still an open question. Developing a full picture understanding of these relationships requires data on organizational hierarchy (distinguishing managerial and production layers), worker mobility (identifying new hires, internal promotions and separations), human capital (measuring experience and education) and internal managerial policies and processes (measuring management structures). We construct an ideal empirical setting by linking three unique datasets: the full roster of formal employment in Brazil (RAIS), the Brazilian annual industrial survey (PIA) and detailed management practices data from the World Management Survey (WMS). %In this paper, we present a new set of stylized facts that allow for an in-depth comparison of structured management practices at the firm level, manager and worker quality, and their relative contribution to explaining productivity differences across firms. 
 
% We need an "in this paper" statement, but something that sells it a bit better than just "novel set of stylized facts"
 
% 2. Data requirements to develop the understand and our unique setting -- I'm not convinced this should come in the first page. I'd rather have the research question front and centre, first.

%construct an empirical setting that uniquely meets these requirements by linking employer-employee matched data from Brazil's \emph{Rela\c{c}\~{a}o Anual de Informa\c{c}\~{o}es Sociais} (RAIS) to the Brazilian manufacturing firms covered in the World Management Survey (WMS). RAIS is effectively an annual census of all formal-sector jobs in Brazil, which annually records an extensive range of establishment, job and worker characteristics. Initiated in 2002, the WMS project generates consistent, empirical measures of management practices in establishments across countries (\citep{Bloom:NewEmpirical:WP:2014}).  Brazil is one of the thirty-five countries included and a sample of its manufacturing firms were surveyed in 2008 and 2013. 

% 4. Research questions and findings
We exploit our linked sample to characterize the importance of personnel management for workforce composition. We use the RAIS panel between 2003-2013 to estimate the Abowd-Kramarz-Margolis (henceforth AKM) decomposition of log wages into components associated with time-varying worker characteristics, a time-invariant firm effect and a time-invariant worker effect \citep{Abowd1999}. Our primary interest is in the worker-specific component of pay, which we treat as a measure of worker quality. 

We first document the positive relationship between firm productivity and worker quality in Brazil: conditional on all other factor inputs and management practices, firms with higher revenues employ higher quality workers. This is in line with the findings in \citet{Bender:Management:NBER:2016} for Germany. 
% Although in Brazil managers comprise only four percent of the workforce in the typical firm, their effect on sales is twice as large as that of production workers. It is striking how similar the overall patterns are, given the differences between Brazilian and German industries. Furthermore, over time, firms in the top tercile of WMS management scores capture a larger employment share at the expense of the bottom tercile of scores. 
% We build on work from \citet{Bender:Management:NBER:2016} using German data, where, absent occupation information, they classify workers as managers if they placed in the top quartile of the pay distribution. Because we can more cleanly identify managers and production workers based on detailed occupation codes, we can explore the distribution of workers across the full wage distribution between occupations and firms and develop a better understanding of the mechanisms. 
%\footnote{In German WMS sample, managers account for an average of 8.5\% of the firm's workforce.}
Having established that there is indeed a relationship between our measure of worker quality and productivity, we focus on the role personnel management plays in building a productive workforce via hiring, retention, and dismissal practices. We show that firms with structured personnel management practices employ a greater share of high-quality workers than firm that lack structured practices. Furthermore, the incumbent managers in structured firms are twice as likely to be from the top quintile of the manager quality distribution and their advantage among incumbent production workers is almost as great. This advantage is consistently reinforced in hiring and firing. The median manager-level hire in a firm with structured management practices comes from the 60th percentile of the distribution of all hires; firms with unstructured practices get their median manager-level hire from the 48th percentile. The pattern is similar when considering production workers. Firms with structured management hire disproportionately high-quality production workers, while firms with unstructured management practices seem to hire almost at random. Firms with structured management are also much less likely to fire workers at all. When they do, they fire more selectively with respect to worker quality. 

% Insert finding about relative employment growth?

%We aggregate our worker-quality measure to the firm level, distinguishing managers from production workers, combine it with the estimated firm effects and firm-year demographic characteristics, and create a firm-level panel we can match to the WMS. 

%% QUESTION: What happens to the workers who move from labourer to management, or from management to other? Are we only using observations that are managers throughout or workers throughout?

Finally, we find that managerial practices relating to personnel management are relatively more important in explaining variation in production worker quality, while practices relating to operations management are relatively more important for explaining variation in manager quality. For both managerial and production workers, however, the key personnel practice linked to worker quality is one that measures whether firms have policies that expressly make hiring high-quality workers a top priority for the firm, and reward senior managers accordingly. 

% 5. Position in the literature/related studies
Our paper bridges the literature in organizational economics studying how firms recruit and retain workers, with the literature in labor economics focused on the role of worker sorting across firms. The labor literature highlights the importance of worker-firm matching for understanding inequality \citep{Card2013,Alvarez:Firms:AEJM:2018,Song:Firming:QJE:2018}, gender wage gaps \citep{Card:Bargaining:QJE:2016}, compensating differentials \citep{Lavetti:CDEM:WP:2018,Sorkin:Ranking:QJE:2018}, and productivity \citep{Iranzo2008}. While these studies generally abstract away from the process by which firms actually assemble their workforce, \citet{Hensvik2018} show that firms actively manage workgroups to retain high output, and \citet{shaw_bosses} estimate that managers and supervisors substantially affect the productivity of their teams. Further, there is evidence that turnover is costly: \citet{Jager2016} and \citet{Gallen2018} show evidence that this is the case because it is difficult to replace incumbent workers with workers from the external labor market. This empirical work is consistent with the recent emphasis in organizational economics on understanding the value of recruiting and retention practices \citep{OyerSchaefer:HLE:2011,Hoffman:Discretion:QJE:2018}. 

The empirical connection between overall management practices and firm productivity has been well-established \citep{bloom_india2012, wms_jeea}. However, despite a large conceptual and theoretical body of work on ``what managers do'' \citep{gibbonshenderson_2012}, we still have little empirical evidence on whether and how structured personnel management practices affect workforce quality. Our paper is most closely related to  \citet{Bandiera:Matching:JOLE:2015} and especially \citet{Bender:Management:JOLE:2018}. Both papers combine survey data on firm management practices with administrative data that can measure hiring and compensation outcomes. \citet{Bandiera:Matching:JOLE:2015} use Italian administrative data for a random sample of service sector executives to show a rich characterization of how managers and firms match through high- and low-powered incentives. \citet{Bender:Management:JOLE:2018} link WMS data for German firms to administrative earnings records to study how management practices and employee ability are related to productivity. Our work is also related to \citet{Friedrich2017}, who shows that more productive firms tend to hire managers internally and also select high-quality candidates in terms of their human capital and skill levels. However, none of the papers in the literature focus directly on personnel management, nor are able to directly distinguish managerial and non-managerial workers and characterize the relationship between these factors as we are able to do with our unique empirical setting.

% I'm really not sure how to weave this in now. \footnote{There is some evidence that better management is associated with more efficient use of energy inputs \citep{Boyd:Evidence:JEEM:2014}, though not much on other types of inputs such as labor inputs.} 
% There is a large literature related to personnel policy and productivity, though there is still a gap in understanding how structured management practices may help alleviate the information and agency problems inherent in the labour matching process. \citep{Oyer and Schaeffer 2011} We build on earlier work by \citet{Bender:Management:NBER:2016}, where they match WMS data for German firms to earnings collected by Germany's Institute for Employment Research and examine how management practices and employee ability are related to productivity. They find that better managed firms are more productive, but a significant portion of the effect of management quality is explained by the human capital of the highest-paid workers in the firm. Two important limitations of the German data, however, are that their records do not include occupation codes (to cleanly distinguish between production workers and managers) nor the reasons for separation (to distinguish fires from quits). Using detailed Danish data, \citep{Friedrich2017}  exploits job-to-job transitions and occupational switching and finds that more productive firms tend to hire managers internally and select high-quality candidates in terms of their human capital and skill levels. The Danish setting, however, lacks data on the level of adopted managerial structures and also does not record the reason for a job separation. 

% \citep{Bender:Management:NBER:2016} % ***UPDATE TO JOLE CITE

% \textbf{This paragraph is clunky. I'm leaving it for Ian to do his magic with lit knowledge powers.}
%  Our work also adds to the literature on sorting and matching. There is evidence that assortative matching is a key driver of inequality in the US, Brazil and Germany, \citep{???} though it is unclear what the specific mechanisms behind this assortative matching are within and across firms. There is evidence that more productive (and better-paying) firms employ more high quality (generally better paid) workers. \citep{CHK; Engbom and Moser; Alvarez et al.} There is also evidence that workers sort both on the basis of wage as well as non-wage aspects of the job \citet{Sorkin; Lavetti and Schmutte; Hotz et al.}, though we rarely observe non-wage aspects of the job (note, here is where we could do a nod to our result on operations management explaining variation in the manager quality).  

% [Need a ``finishing sentence'' or two introducing our story]

% ***ADD Friedrich TO BIBTEX FILE 

%% NEED TO INCLUDE A LOT MORE CITATIONS HERE TO REALLY PINPOINT WHAT WE CAN LEARN THAT IS NEW AND DIFFERENT RELATIVE TO WHAT IS ALREADY KNOWN. 

% 6. Broader implications 


