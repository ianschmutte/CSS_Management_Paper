% !TeX root = ./AER_Insights.tex

This paper studies the role of management in assembling a high quality and productive workforce. On this topic, existing empirical research provides two different kinds of evidence, the first based on surveys about the practices of plant managers, and the second based on the observed hiring, firing, and compensation outcomes of individual firms. 
In the World Management Survey, for example, managers report whether they have formal practices in place to attract and retain skilled workers, to create incentives for high-performing workers, and for identifying and removing their less effective employees \citep{Bloom2012}. 
By contrast, most studies based on observational data do not know about the practices used by management, and instead study patterns in data on pay and job transitions to infer the nature of decision-making within the firm \citep[e.g.][]{AbowdKramarzRoux:EJ:2006,Hensvik2018}.
It is natural to surmise that the inferred behaviors drawn from observational data reflect the implementation of formal management practices described in the survey-based literature. This need not be the case. Many models of the labor market illustrate that different firms may optimally attract workers of differing ability, without reference to specialized investments in managerial processes, nor in managerial talent.\footnote{The assortative matching model of \citet{becker1973theory} is the classic example, but in most contemporary discussions of job matching in economics, like \citet{Card:FirmsIneq:JOLE:2017} or \citet{Sorkin:Ranking:QJE:2018}, matching happens without the intervention of any particular or specialized managerial competence.} It is clear that labor market models abstract away from real-world differences in the personnel management practices of different firms. It is less clear how much those differences matter, and why.

We combine the approaches in the prior literature to assess the relationship between a firm's stated use of what we call \emph{structured} management practices and its observed success in hiring and retaining high-quality workers. Despite a large conceptual and theoretical body of work on ``what managers do'' \citep{gibbonshenderson_2012}, there is still little empirical evidence on how the practices employed by managers affect workforce quality. Structured management practices must be implemented by skilled managers, %insert citation
 and so we focus in particular on the quality of both the managerial and non-managerial workforce. We find that firms using structured management practices are indeed more successful, both at recruiting and keeping better managers, as well as better production workers. We also provide evidence that these findings reflect behavioral differences between firms that use structured management practices and those that do not.

Our analysis is based on a unique dataset that links firm-level survey measures of management practices from the World Management Survey (WMS) to matched employer-employee data from Brazil, the \emph{Rela\c{c}\~{a}o Anual de Informa\c{c}\~{o}es Sociais} (RAIS). The WMS data provide consistent measures of the management practices in a representative repeated cross-section of manufacturing establishments in Brazil (along with thirty-four other countries, \citet{bloom_qje2007}. 
% The WMS interview covers 18 basic management practices, a third of which relate specifically to the ``people'' or personnel management practices that directly target hiring, retention and dismissal decisions. 
Based on these data, we characterize each firm as having either structured or unstructured management practices, with respect to both its management of people, and its management of production operations. We match these management statistics to a specific measure of the quality of each establishment's managerial and non-managerial workforce derived from the RAIS. The longitudinal structure of RAIS lets us measure each firm's success in using hiring, firing, and retention to maintain a high-quality workforce.

%In Section \ref{}, we introduce our data and describe how we measure our key concepts, namely structured management practices, and worker quality. There, we also describe how we use establishment-level productivity data to validate our measure of worker quality. In Section \ref{}, we present our main findings. 
Our findings can be summarized as follows. First, firms with structured personnel management practices hire disproportionately from the top half of the worker quality distribution, especially when recruiting managers. Second, these firms are also more successful at keeping their high-quality hires. Their incumbent managers are twice as likely to be high-quality and their advantage among incumbent production workers is almost as great. Third, firms with structured management practices have much lower overall rates of quits and fires, suggesting they screen or motivate workers more effectively. Furthermore, when they do dismiss workers, firms using structure practices fire more selectively with respect to worker quality.  Finally, we show that the most important personnel practice for attracting and retaining a high-quality workforce is ``instilling a talent mindset'', where hiring high-quality workers is expressly a top priority for the firm and senior managers are rewarded accordingly.  This is true for both production workers and managers. However, we also find that operations management practices are relatively more important for explaining variation in manager quality, suggesting that well-defined targets, accountability and performance evaluation facilitate matches with better managerial talent. 

The key takeaway is that the way workers match to firms depends on the quality of each firm's management practices and the allocation of managerial talent. Our results thus speak to a need for further integration between research on the role of firms in labor markets and research on organizational behavior focused on recruiting and retention. A handful of other papers have arrived at similar conclusions. Similar to our paper, \citet{Bender:Management:JOLE:2018} document a positive relationship between firm productivity and worker quality using WMS data linked to German administrative records. Unlike them, we are able to distinguish managerial workers directly, and can isolate the different interactions between management practices and managerial versus non-managerial workers.
\citet{Bandiera:Matching:JOLE:2015} use Italian administrative data to explore how managers and firms match through high- and low-powered incentives.  Finally \citet{Hoffman:Discretion:QJE:2018} shows that good management practices are needed to overcoming managerial biases in hiring decisions.

Taken together, the literature suggests that the institutional environment in the firm, manifest in its management practices, cannot be reduced to the quality of its managers or its workforce. 
The labor literature highlights the importance of firms and worker-firm matching for understanding inequality \citep{Card2013,Alvarez:Firms:AEJM:2018,Song:Firming:QJE:2018}, gender wage gaps \citep{Card:Bargaining:QJE:2016}, compensating differentials \citep{Lavetti:CDEM:WP:2018,Sorkin:Ranking:QJE:2018}, and productivity \citep{Iranzo2008}. 
These findings suggest that firms' decisions about who to hire and fire affect not only their own bottom line, but the functioning of the entire labor market. 
However, these studies generally abstract away from the processes firms use to assemble their workforce.
\citet{OyerSchaefer:HLE:2011}, suggest that economists should focus on the active role played by management in recruiting the best workers. Recent research based on observational data points to the importance of good managers for team production \citep{shaw_bosses,Hensvik2018}, managing the cost of turnover \citep{Jager2016,Gallen2018}, and
the selection of skilled workers \citep{Friedrich2017}. 
Therefore, understanding how firms interact with and affect labor markets can and should be further investigated as data on management practices and linked employer-employee data become more widely available.




 

% However, each of these papers lacks some important element of our unique empirical setting -- either the ability to distinguish managers from production workers, identify the reason for separation or empirically measure managerial practices.



% Eeckhout and Kircher (2011) 