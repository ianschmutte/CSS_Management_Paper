%TCIDATA{LaTeXparent=0,0,noname.tex}

%-------------------- Packages --------------------
    \usepackage{booktabs}
    \usepackage{epstopdf}
	\usepackage{fancyref}
	\usepackage{array}
	\usepackage{hyperref}
	\usepackage{graphics}
	\usepackage{float}
	\usepackage{subcaption}
	\usepackage{caption}
	\usepackage{amsfonts}
	\usepackage{amsmath}
	\usepackage{natbib}    % for bibliography
	\usepackage{amssymb}  % AMS symbols
	\usepackage{amsfonts}
	\usepackage{tikz}		%for nice-looking graphs
	\usepackage{pgfplots}	%for nice-looking plots of data (requires tikz package)
		\pgfplotsset{compat = newest} %to use newest plot settings in pgfplot
	\usepackage{graphicx} % for inclusion of EPS graphics
	\usepackage{multicol} % Multicolumn formatting (in-line tables)
	\usepackage{rotating}  % allows for rotated tables
	\usepackage{setspace}
	\usepackage{multirow} %allows multirows/cols in tabls
	\usepackage{array}     % Modifications of tabular environment
	\usepackage{threeparttable} %Allows tables with automatic notes section
	\usepackage{longtable} % allows for tables to wrap across pages.
	\usepackage{supertabular} % allows for tables to wrap across pages.
	% \usepackage{subfloat}	%to put multiple floats in the same frame
	\usepackage{color}
	\usepackage{lscape}
	\usepackage{geometry}
	%\makeindex
	\usepackage{dcolumn}
	\newcolumntype{k}{D{.}{.}{1.3}}
	\newcolumntype{d}{D{.}{.}{2.2}}
	\newcolumntype{s}{D{.}{.}{1.0}}
	\newcommand{\m}[1]{\multicolumn{1}{c}{#1}} %for use with dcolumn
	\newcolumntype{f}{D{.}{.}{2.4}}

%-------------------- hyperref setup-----------------
	% defining colors for package hyperref
	\definecolor{myblue}{rgb}{0,.2,1}

	\hypersetup{%
	%backref=true,%
	naturalnames=true,%
	bookmarksnumbered=true,%
	bookmarksopen=false,%
	plainpages=true,%
	colorlinks=true,%
	urlcolor=myblue,
	linkcolor=myblue,%
	filecolor=myblue,%
	citecolor=black,%
	%pagecolor=myblue,%
	%pdftitle={\myshorttitle},%
	%pdfpagemode=UseOutlines,%
	%pdfauthor={\myshortauthors},%
	%pdfsubject={\myshorttitle}
	}%

%----------------------
	%To insert un-numbered footnotes (as on the title page)
	%ref: http://en.wikibooks.org/wiki/LaTeX/Formatting#cite_note-csli_footnotes-0
	\makeatletter
	    \def\blfootnote{\xdef\@thefnmark{}\@footnotetext}
	\makeatother
	% http://tex.stackexchange.com/questions/8351/what-do-makeatletter-and-makeatother-do


%-------------------- choose the font here --------------------
	\usepackage{times}
	%\usepackage{newcent}
	%\usepackage{helvet}
	%\usepackage{helvetic}
	%\usepackage{ncntrsbk}
	%\usepackage{bookman}
	%\usepackage{avantgar}
	% \usepackage{palatino}

%-------------------- formatting of fancy references --------------------
	%\newcommand{\Cite}{\citeasnoun} % this for harvard
	\newcommand{\Cite}{\citet} % this for natbib
	\bibpunct{(}{)}{;}{a}{}{;} %set if using natbib

%-------------------- Common Math Operators ----------
	\DeclareMathOperator{\Corr}{Corr}
	\DeclareMathOperator{\cov}{cov}
	\DeclareMathOperator{\var}{var}
	\DeclareMathOperator{\E}{E}
	\DeclareMathOperator{\F}{F}
	\DeclareMathOperator{\f}{f}
	\DeclareMathOperator{\J}{J}
	\DeclareMathOperator{\Tp}{T}

%-------------------- Special Commands
	\newcommand{\s}{^{*}} %for putting stars in tables
	\renewcommand{\ss}{^{**}}
	\newcommand{\sss}{^{***}}
	\newcommand{\fnone}{\textsuperscript{1}}
	\newcommand{\fna}{\textsuperscript{a}}
	\newcommand{\fnb}{\textsuperscript{b}}
	\newcommand\T{\rule{0pt}{2.6ex}}         %for manipulating space in tables
	\newcommand\B{\rule[-1.2ex]{0pt}{0pt}}	 %for manipulating space in tables
	\newcommand\C{\rule{0pt}{0pt}}
	\newcommand\mrt{\multirow{2}{*}}

%-------------------- Common Theorem Environments ------------------------
	\newtheorem{theorem}{Theorem}
	\newtheorem{algorithm}{Algorithm}
	\newtheorem{axiom}{Axiom}
	\newtheorem{case}{Case}
	\newtheorem{claim}{Claim}
	\newtheorem{conclusion}{Conclusion}
	\newtheorem{condition}{Condition}
	\newtheorem{conjecture}{Conjecture}
	\newtheorem{corollary}{Corollary}
	\newtheorem{criterion}{Criterion}
	\newtheorem{definition}{Definition}
	\newtheorem{example}{Example}
	\newtheorem{exercise}{Exercise}
	\newtheorem{lemma}{Lemma}
	\newtheorem{notation}{Notation}
	\newtheorem{problem}{Problem}
	\newtheorem{proposition}{Proposition}
	\newtheorem{remark}{Remark}
	\newtheorem{solution}{Solution}
	\newtheorem{summary}{Summary}




%%% End: